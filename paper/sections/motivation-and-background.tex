\section{Motivation and Background} \label{s:motivationAndBackground}
Convolutional Neural Networks (CNNs) have changed the machine learning landscape by providing numerous approaches to solving various problems. This technique can be used to deal with a great amount of data and is widely used in many domains such as image recognition, vision, and so on. However, the CNN model can be easily manipulated by adversarial examples. An adversarial example is a input specially crafted to cause deep neural networks to misclassify with minimum changes to the input pixels \cite{adversarial}. This adversarial input can hardly be detected by human eyes, though can force a well-trained CNN model to misclassify. There are many kinds of algorithms for generating adversarial examples, some of the most popular tools are listed in the Cleverhans library \cite{papernot2018cleverhans}. A basic attack algorithm can generate adversarial examples based on the knowledge of the structure of the CNN model and make it misclassify, thus it is called a white-box attack. Furthermore, an attacker may simulate the dataset of a CNN model and create an adversarial attack without knowing the details of the machine learning model, which is called black-box attack \cite{blackbox}.

Since the model are not robust enough to prevent these attacks, our motivation is to design a defense method to protect our model. To build such a model, we must consider the trade-off, because if a defense model greatly reduces the accuracy for non-adversarial examples, it is not practical even if its performance is perfect when facing an adversarial attack. Filters are widely used to prepossess images and can achieve some functions such as denoising. The perturbation of adversarial examples is somewhat like a kind of noise added to the original images, so our idea is to use filters to deal with adversarial inputs and gain a satisfying trade-off. Some papers point out that a single filter is not good enough, but a robust model can be built based on simple filters \cite{ComplexFilter}. So, we want to develop the simple filters and build a better model. Since filters can be combined together, we are going to build an ensemble filter model and test its function.
